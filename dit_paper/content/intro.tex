\newpage
\section{Introduction}
Blockchain technologies are one of the key technologies that can change the Internet of today. In the last decade we have witnessed the proliferation of a vast amount of blockchain networks and protocols \cite{}, tailored to fit the requirements of various applications ranging from payment transactions to traceability of goods.
%
The common denominator of blockchain networks is the adaption of the computing paradigm of decentralization. The paradigm is the natural evolutionary step forward from the distributed computing paradigm introduced in the era of cloud computing to a more autonomous, open and democratically governed one. This result is a liberalization of computing systems: No entity can game the system of autonomously operating nodes as long as a democratic majority exists. Surprisingly the openness and decentralized trust blockchain technologies allow to realize on a technological level complies with the social-economical change introduced with globalization \cite{}. It seems the blockchain paradigm is the technical tool to meet the demands of a globally interconnected society.





\subsection{Problem Statement}